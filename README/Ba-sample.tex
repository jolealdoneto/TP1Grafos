\documentclass[a4paper,11pt,oneside]{article}

% $Id: ba-sample.tex 952 2006-03-06 20:14:35Z schoenw $

% To use this template, you have to have a halfway complete LaTeX
% installation and you have to run pdflatex, followed by bibtex,
% following by one-two more pdflatex runs.
%
% Note thad usimg a spel chequer (e.g. ispell, aspell) is generolz
% a very guud ideo.

\usepackage{times}		%% native times roman fonts
\usepackage{parskip}		%% blank lines between paragraphs, no indent
\usepackage[pdftex]{graphicx}	%% include graphics, preferrably pdf
\usepackage[pdftex]{hyperref}	%% many PDF options can be set here
\pdfadjustspacing=1		%% force LaTeX-like character spacing

\hypersetup{
  pdfauthor = {Jos\'e Leal Domingues Neto},
  pdftitle = {Entreg\'avel TP1},
  pdfkeywords = {tp1}
}

\begin{document}

  \thispagestyle{empty}

  \begin{flushleft}
    \textbf{\huge TP1 - Grafos}
  \end{flushleft}
  \vspace*{6mm}
  \begin{flushleft}
    \textbf{\large Jos\'e Leal Domingues Neto}\\[2ex]
  \end{flushleft}
  \vspace*{1mm}
  \begin{flushleft}
    \textit{
      Date: \today \\
      PAA \\
      DCC - UFMG}
  \end{flushleft}
  \vspace*{2mm}

\newpage

  \section{Algoritmos}
    \subsection{BFS}
    O BFS foi implementado de acordo com o algoritmo do livro. Um detalhe e dificuldade em particular foi como implementar o hist\'oarico de arestas visitadas. Como esse dado n\~ao \'e normalmente dado pelo algoritmo, foi implementado um Map{\textless}Node, List{\textless}Edge{\textgreater}{\textgreater} onde para cada n\'o a aresta visitada era adicionada, justamente com todas as arestas visitadas do n\'o anterior. O BFS est\'a implementado na BFS.java, e implementa a interface AllPairs.
    
    \subsubsection{Big-O}
    Runtime: \begin{math}O(V + E)\end{math} para o BFS, \begin{math}O(V)\end{math} para cada n\'o, o que \'e \begin{math}O(V * (V + E))\end{math}. Mas como temos \begin{math}O(E)\end{math} dentro do for para procurar a aresta de menor peso, temos \begin{math}O(V * (V + 2*E))\end{math}. \\
    =\textgreater \begin{math}O(V*E)\end{math} \\\\
    Mem\'oria: Come\c{c}amos com \begin{math}O(V+E)\end{math} para lista de adjac\^encia. Porem adicionalmente guardamos os caminhos, que \'e \begin{math}O(V * E)\end{math}. \\
    =\textgreater \begin{math}O(V * E)\end{math}
    
    \subsection{Repeated Squaring}
    A parte te\'orica foi baseada no livro, com m\'inimas modifi\c{c}\\~oes na implementa\c{c}\~ao em si. \'E utilizada a matrix de adjac\^encia para guardar e calcular o RS. Para guardar os caminhos m\'inimos, utiliza-se o uma matrix de listas List{\textless}Edge\textgreater[][] para guardar para cada \textbf{i} e \textbf{j} no algoritmo. Como antes, adiciona-se todas as arestas do n\'o anterior mais a seguinte. Ela implementa a interface AllPairs e extende a classe AbstractAdjMatrixTraverse que \'e dividida com o Floyd-Warshall.
    \subsubsection{Big-O}
    Runtime: \begin{math} O(V^4) \end{math} para o RS. \\
    =\textgreater \begin{math}O(V^4)\end{math} \\\\
    Mem\'oria: Come\c{c}amos com \begin{math}O(V^2) \end{math} para a matrix de adjac\^encia. Porem adicionalmente guardamos os caminhos para cada \textbf{i} e \textbf{j}, que \'e \begin{math}O(V^2 * E)\end{math}. \\
    =\textgreater \begin{math}O(V^2 * E)\end{math}
    

    \subsection{Floyd-Warshal}
    Para o Floyd-Warshall muito da metodologia foi reutilizada do RS para implementar o algoritmo. J\'a que tudo como matrix de adjac\^encia j\'a havia sido feito, foi criado uma classe que os dois compartilharam a AbstractAdjMatrixTraverse. Com o Floyd-Warshall, tamb\'em foi usado uma matrix de arestas para cada \textbf{i} e \textbf{j}.
    \subsubsection{Big-O}
    Runtime: \begin{math} O(V^3) \end{math} para o Floyd-Warshall. \\
    =\textgreater \begin{math}O(V^3)\end{math} \\\\
    Mem\'oria: Come\c{c}amos com \begin{math}O(V^2) \end{math} para a matrix de adjac\^encia. Porem adicionalmente guardamos os caminhos para cada \textbf{i} e \textbf{j}, que \'e \begin{math}O(V^2 * E)\end{math}. \\
    =\textgreater \begin{math}O(V^2 * E)\end{math}
    
    \subsection{Johnson}
    O algoritmo de Johnson \'e complexo por haver a necessidade de implementar v'arios algoritmos em um s\'o. Ou seja, foi implementado o Dijkstra juntamente com Bellman-Ford que n\~ao possuem f\'acil l\'ogica. Para aumentar a complexidade, h\'a a necessidade de repesar os pesos dos n\'os ap\'os o Bellman-Ford. Para sua implementa\c{c}\~ao O j foi usada uma lista de adjac\^encia contando os caminhos encontrados em um mapa Map{\textless}Node, List{\textless}Edge{\textgreater}{\textgreater} para cada n\'o. Ap\'os isso, a conta das arestas utilizadas \'e feita e o betweeness \'e determinado.
    \subsubsection{Big-O}
    Runtime: \begin{math} O(V^2 lg V) \end{math} para o Johnson com HEAP, \begin{math} O(V^3) \end{math} para o Johnson com List,  \\
    =\textgreater \begin{math}O(V^3)\end{math} \\\\
    Mem\'oria: Come\c{c}amos com \begin{math}O(V+E)\end{math} para lista de adjac\^encia. Porem adicionalmente guardamos os caminhos, que \'e \begin{math}O(V * E)\end{math}. \\
    =\textgreater \begin{math}O(V * E)\end{math}
    
    \section{Johnson vs. Floyd-Warshall}
    Os dois algoritmos cumprem o mesmo trabalho, por\'em o algoritmo de Johnson \'e mais recomendado quando se lida com grafos do tipo sparse. Como pode-se usar um algoritmo mais r\'apido para determinar o menor caminho nos casos de poucas arestas, evita-se o uso do Floyd-Warshall.
    
    \section{Tempo de execu\c{c}\~ao e mem\'oria}
    \subsection{Tempo de execu\c{c}\~ao}
    Alguns testes foram feitos, e temos os seguintes resultados.\\\\
    \includegraphics[width=1.1\textwidth]{../res/dolphins-40.jpg}\\
    Este exemplo foi rodado em um grafo teoricamente pequeno o dolphin.gml. Ele possui por volta de 100 n\'oas.\\
    Como podemos ver, o algoritmo de Repeated Squaring possui um tempo de execu\c{c}\~ao muito maior do que os outros. Mesmo neste exemplo, onde o eixo Y \'e dado em milisegundos, vemos que j\'a se torna imposs\'ivel rodarmos o algoritmo com efici\^encia. Temos a seguir um exemplo com um tempo de execu\c{c}\~ao maior:\\
    \includegraphics[width=1.1\textwidth]{../res/adjnoun-90.jpg}\\
    Este exemplo foi rodado no grafo adjnoun.gml, que \'e um pouco maior, contendo 110 n\'oas.\\
    Como vemos, o RS rodou por mais de 3 minutos, enquanto os outros algoritmos demoraram menos de 30 segundos.
    \subsection{Mem\'oria}
    Muito se foi aprendido com este experimento. Foi visto que na verdade em cada algoritmo se troca muitas vezes espa\c{c}o por velocidade. O jconsole do Java foi utilizado para apurar estes dados.\\\\
    \includegraphics[width=1.1\textwidth]{../res/bfs-dolphin8.jpg}\\\\
    Aqui foi rodado um BFS com um o grafo dolphin.gml. A mem\'oria utilizada foi pouca e o algoritmo foi performante. \\\\
    \includegraphics[width=1.1\textwidth]{../res/floyd-dolphins-8.jpg}\\\\
    Aqui vemos o algoritmo de Floyd-Warshall rodando sob as mesmas condi\c{c}\~oes. Como se v\^e ele necessita de quase o dobro da mem\'oria do anterior. \\\\
    \includegraphics[width=1.1\textwidth]{../res/johnson-dolphin-8.jpg}\\\\
    E por fim este \'e o Johnson. V\^e se claramente que apesar do algoritmo de Johnson ser mais r\'apido, ele faz uma troca por espa\c{c}o.\\
    
    \subsection{Ambiente testado}
    O Ambiente testado foi em um Dell XPS13, Intel i7 Haswell com 8Gb de mem\'oria.
\end{document}
